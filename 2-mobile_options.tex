\chapter{Understanding Mobile Application Development Options}
\justifying
\section{Introduction}

This chapter provides a detailed explanation on the distinct types of mobile applications that have been already introduced. As a result, the reader will be aware of the pros and cons that every scenario involves. Special attention will be payed to hybrid mobile applications, not only because the whole project is based on them, but also because they are the less well-known.

One section will be devoted for each and every scenario, discussing how are they developed and exposing their strengths and weaknesses. A brief summary of the technologies and tools needed for each will also be provided. The chapter will conclude with a table to sum up all the available scenarios.

\section{Native Mobile Applications}

A native mobile application is an application coded in a specific programming language. The most popular are Java for Android and Swift or Objective C for iOS operating systems. They provide the best usability, the best features, the best performance, the best overall user experience, plus are highly reliable. Because of this, apps like video games tend to be developed this way, as they are resource intensive and a bad performance could hurt the user experience. However, this type of app is tied to one single operating system, forcing the development team to make duplicate versions that work on other platforms. Therefore this may not be the best option for apps intended to spread throughout the whole market.

Their key features are:

\begin{itemize}
	\item \textbf{Multi touch handling:} native apps support all kinds of multi touch events, from double taps or pinches to pressure-sensitive taps.
	\item \textbf{Built-in components:} camera, GPS or encrypted storage are only a few of the features native to a device that can be smoothly integrated into an app. Built-in components are of such importance to some apps that may be a deciding factor on which mobile technology one should choose.
	\item \textbf{Fast graphics and fluid animations:} for those apps which are using a lot of data and require a fast refresh, are highly interactive or make use of intensely computational algorithms, native apps provide the highest efficiency and the best results.
	\item \textbf{High reliability:} the abstraction while coding is lower than on the other types of mobile applications. Therefore, as a developer, you got more control over what is going on inside your app, and nasty surprises are less likely to happen. 
	\item \textbf{Improved user experience and familiarity:} people are accustomed to the native platform. Making use of all the native features will be easier for the average user and the app will be just plain easier to use.
	\item \textbf{Documentation:} apart from the official docuementation, there are thousands of books for Android and iOS development, and many more blogs, articles and websites with detailed information about every feature and little crevice.
\end{itemize}

After taking a look at the pros of developing a native app, it is clear that they are unbeatable on some aspects. However, they suffer from a number of issues:

\begin{itemize}
	\item \textbf{System restricted:} the fact that native apps are developed using specific programming languages turns porting an app between platforms into a time-consuming task because most of the code will be simply useless. 
	\item \textbf{Difficult to develop:} the programming languages used are not the easiest ones, so simply cutting and pasting Objective-C or Java will neither suffice nor work.
	\item \textbf{High level of experience required:} because native apps are difficult to develop, the level of experience required is higher than other development scenarios. Indeed, the technological know-how of a developer is a very important consideration.
	\item \textbf{Require an integrated development environment:} an integrated development environment, or IDE, provides tools for building, debugging, managing, version control and some other critical tools that professional developers need. These tools are needed because of the difficulties involved in the development process. In fact, usually different IDEs are used for each programming language, and even though they tend to be very similar to one another, so they become another thing to add to the list of inconvenients.
\end{itemize}

After considering all the pros and cons concerning the development of a native mobile application, we can conlude that it is the best way to develop ambitious, compute-intensive, single platform mobile applications.

\section{Web Mobile Applications}

A web mobile application, also known as HTML5 mobile application, is basically a web page designed to work on the small screen of a smartphone. This makes them compatible with every modern mobile browser. Developing them is easier as well: the technological bar and the learning curve of website programming languages are singnificantly lower than those of Objective C, for example.

Web apps most remarkable aspects are the following:

\begin{itemize}
	\item \textbf{Write once, run anywhere:} web applications take this concept to the extreme. As long as the device is running a modern browser, the user can use the app. This makes sure that it will reach the whole market. 
	\item \textbf{Searchable:} because the content is on the web, it can be searched, which can be a huge boost to its popularity.
	\item \textbf{Easy to develop:} the complexity involved in developing a web page is usually diminished by the fact that is easier to understand and of a higher level than native apps programming languages.
	\item \textbf{Distribution and support:} native mobile applications are dependent upon marketplaces for their distribution and support. Nevertheless, web applications are hosted on a server kept under control by the development team, so fixing a bug or adding features becomes a much easier task.
\end{itemize}

