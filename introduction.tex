\chapter{Introduction}
\justifying
\section{Context}
Not long ago, before developing a mobile application, or app, one of the most important decisions that had to be taken was choosing the platforms that your app would run on. Because a different programming language is used for each platform, the same application would have to be developed twice, thrice or even more times. In fact, the amount of work involved could be such that would limit the efforts of the development team to a single platform.

Along with mobile applicationss, another way of adapting web services for smartphones emerged: web mobile applications. This type of applications use standard web technologies, generally HTML5, CSS and JavaScript, to adapt web pages and make them suitable for the smaller screens of smartphones and tablets. Though this concept saves time, some vital limitations remain, mostly access to native device functionalities such as camera or GPS.

Aiming to increase the productivity and efficiency of a development process, the concept of hybrid mobile applications was born. Through an Aplication Programming Interface, or API, it is possible to embed HTML5 apps inside a native container, so the same code can be ported across systems. Hybrid mobile applications combine the best elements of native and HTML5, and therefore represent a new stage in the evolution of mobile applications development.

In this context, currently the most popular technology for this purpose is Adobe PhoneGap. However, some of its own peculiarities make PhoneGapapps inefficient and slow, and somehow feel less polished  than a native app. In an attempt to fix these issues and challenge PhoneGap's dominant position, React Native came on the scene on April 2015. React Native's main feature is that it uses native components to achieve a completely native application user experience.

\section{Project goals}

This project aims to provide an explanation on two main topics: firstly, the differences betweeen  PhoneGap and React Native, and secondly, a test scenario to help understand the most important features that revolve around React Native.

Among the main goals inside this project, we can find:

\begin{itemize}  
	\item Deepen the knowledge on the varios types of mobile applications.
	\item Introduce React and some of its key features.
	\item Understanding what React Native is and how it works.
	\item Design and build a simple test scenario to demonstrate React Native's capabilities.
\end{itemize}

\section{Structure of this document}

This project is divided in X chapters, arranged following a general-to-specific pattern. The structure is as follows:
