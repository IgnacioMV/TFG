\chapter*{Resumen}
\addcontentsline{toc}{chapter}{Resumen}
\justifying
\pagenumbering{Roman}
\setcounter{page}{5}

La aparición de los smartphones ha propiciado la proliferación de multitud de aplicaciones móviles para todas las plataformas existentes. En las distintas tiendas de aplicaciones podemos encontrar desde apps para acceder a las redes sociales o juegos, hasta algunas orientadas a profesionales de distintos sectores. Esto ha dado lugar a un creciente mercado dedicado al desarrollo de las mismas empleando lenguajes de programación como Java, Objective C o Swift. 

Sin embargo, la existencia de plataformas muy distintas y propiedad de diferentes compañías ha provocado que una vez desarrollada una app para una de ellas, sea necesario rehacerla parcial o completamente si deseamos adaptarla a otro sistema. 

Con el objetivo de facilitar la portabilidad entre plataformas surgió el concepto de las aplicaciones móviles híbridas. A través del uso de tecnologías web estándar como son HTML5, CSS o JavaScript, y mediante una API, se pueden desarrollar aplicaciones móviles multiplataforma de forma rápida y eficiente. 

Actualmente existen dos tecnologías para el desarrollo de aplicaciones móviles híbridas: Phonegap, con una ampplia base de usuarios, y React Native, más reciente y en continua actualización. En este Trabajo de Fin de Grado se expondrán las principales diferencias entre ellas y se desarrollará una aplicación con ambas con objeto de poner de relieve dichas diferencias. 

Asimismo se diseñará e implementará un escenario de prueba con React Native, que sirva de introducción a dicha tecnología. Se documentará todo el proceso, desde la instalación de las distintas herramientas necesarias hasta la ejecución de una aplicación en un terminal real.

\vspace{5mm}
\large{\textbf{Palabras clave: } Aplicación Móvil, React, Phonegap, JavaScript.}
\afterpage{\blankpage}