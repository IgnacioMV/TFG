\documentclass[11pt,a4paper]{report}
\usepackage[utf8]{inputenc}
\usepackage[spanish]{babel}
\usepackage{multicol}
\usepackage{multirow}
\usepackage{amsfonts}
\usepackage{amssymb}
\usepackage{float}
\usepackage{ragged2e}
\usepackage[parfill]{parskip}
\usepackage[left=3cm,right=3cm,top=2.5cm,bottom=2.5cm]{geometry}
\usepackage{afterpage}
\usepackage{graphicx}
\usepackage{sectsty}

\graphicspath{{img/}}
\newcommand\blankpage{%
	\null
	\thispagestyle{empty}%
	\newpage}

\newcommand{\pic}[5]{\begin{figure}[#1] %Easy pic **Arguments**: {Position <h, H, t, b>}{Pic width/text width (real) <0 - 1>}{path}{caption}{reference}
		\centering
		\includegraphics[width=#2\textwidth]{#3}
		\caption{#4}
		\label{#5}
	\end{figure}
}
\newcommand{\clpic}[3]{\begin{figure}[#1] %Easy captionless pic **Arguments**: {Position <h, H, t, b>}{Pic width/text width (real) <0 - 1>}{path}
		\centering
		\includegraphics[width=#2\textwidth]{#3}
	\end{figure}
}

\begin{document}
\begin{titlepage}
\pagenumbering{gobble}
\centering
\vspace*{\fill}
\huge{GRADO DE INGENIERÍA DE TECNOLOGÍAS Y SERVICIOS DE TELECOMUNICACIÓN}\\
\vspace*{\fill}
\huge{Desarrollo de una aplicación móvil usando las tecnologías React Native y Phonegap y diseño e implementación de un escenario de prueba con React Native}\\
\vspace*{\fill}
\huge{Ignacio Martín Velasco}\\
\huge{2016}\\
\vspace*{\fill}
\afterpage{\blankpage}
\end{titlepage}
\vspace*{10mm}
\textbf{\large{TRABAJO FIN DE GRADO}\\}
\begin{table}[H]
	\label{my-label}
	\begin{tabular}{lp{10cm}}
		\textbf{TÍTULO:}       & \textbf{Desarrollo de una aplicación móvil usando las tecnologías React Native y Phonegap y diseño e implementación de un escenario de prueba con React Native} \\[2mm]
		\textbf{AUTOR:}        & \textbf{D. Ignacio Martín Velasco}\\[2mm]
		\textbf{TUTOR:}        & \textbf{D. Joaquín Salvachúa}\\[2mm]
		\textbf{DEPARTAMENTO:} & \textbf{Ingeniaría de Sistemas Telemáticos}                                                                                                                     
	\end{tabular}
\end{table}
\vspace*{\fill}
\textbf{\large{TRIBUNAL:  }\\}
\begin{table}[H]
	\label{my-label2}
	\begin{tabular}{lp{12cm}}
		\textbf{Presidente:}       & \textbf{D. Juan Quemada Vives} \\[2mm]
		\textbf{Vocal:} & \textbf{D. Santiago Pavón Gómez}\\[2mm]
		\textbf{Secretario:} & \textbf{D. Gabriel Huecas Fernández-Toribio}\\[2mm]
		\textbf{Suplente:} & \textbf{David Fernández Cambronero}
	\end{tabular}
\end{table}
\vspace*{\fill}
\textbf{\large{FECHA DE LECTURA:\,\,\,\hrulefill}\\[15mm]}
\textbf{\large{CALIFICACIÓN:\,\,\,\hrulefill}\\}
\vspace*{\fill}
\afterpage{\blankpage}

\clearpage

\centering
\fontsize{19pt}{1pt}{\textbf{UNIVERSIDAD POLITÉCNICA DE MADRID}}\\[10mm]
\large{\textbf{ESCUELA TÉCNICA SUPERIOR \\DE INGERNIEROS DE TELECOMUNICACIÓN}}\\[10mm]
\begin{figure}[H]
	\centering
	\includegraphics[width=0.55\linewidth]{logoescuela.jpg}
\end{figure}
\fontsize{19pt}{1pt}{\textbf{GRADO DE INGENIERÍA DE TECNOLOGÍAS Y SERVICIOS DE TELECOMUNICACIÓN}}\\[7mm]
\LARGE{\textbf{TRABAJO DE FIN DE GRADO}}\\[15mm]
\LARGE{\textbf{DESARROLLO DE UNA APLICACIÓN MÓVIL USANDO LAS TECNOLOGÍAS REACT NATIVE Y PHONEGAP Y DISEÑO E IMPLEMENTACIÓN DE UN ESCENARIO DE PRUEBA CON REACT NATIVE}\\[35mm]
\LARGE{\textbf{IGNACIO MARTÍN VELASCO}}\\[5mm]
\LARGE{\textbf{2016}}\\[20mm]
\afterpage{\blankpage}

\clearpage

\newpage
\justifying
\large{\textbf{Resumen}}\\

La aparición de los smartphones ha propiciado la proliferación de multitud de aplicaciones móviles para todas las plataformas existentes. En las distintas tiendas de aplicaciones podemos encontrar desde apps para acceder a las redes sociales o juegos, hasta algunas orientadas a profesionales de distintos sectores. Esto ha dado lugar a un creciente mercado dedicado al desarrollo de las mismas empleando lenguajes de programación como Java, Objective C o Swift. 

Sin embargo, la existencia de plataformas muy distintas y propiedad de diferentes compañías ha provocado que una vez desarrollada una app para una de ellas, sea necesario rehacerla parcial o completamente si deseamos adaptarla a otro sistema. 

Con el objetivo de facilitar la portabilidad entre plataformas surgió el concepto de las aplicaciones móviles híbridas. A través del uso de tecnologías web estándar como son HTML5, CSS o JavaScript, y mediante una API, se pueden desarrollar aplicaciones móviles multiplataforma de forma rápida y eficiente. 

Actualmente existen dos tecnologías para el desarrollo de aplicaciones móviles híbridas: Phonegap, con una ampplia base de usuarios, y React Native, más reciente y en continua actualización. En este Trabajo de Fin de Grado se expondrán las principales diferencias entre ellas y se desarrollará una aplicación con ambas con objeto de poner de relieve dichas diferencias. 

Asimismo se diseñará e implementará un escenario de prueba con React Native, que sirva de introducción a dicha tecnología. Se documentará todo el proceso, desde la instalación de las distintas herramientas necesarias hasta la ejecución de una aplicación en un terminal real.

\vspace{5mm}
\large{\textbf{Palabras clave: } Aplicación Móvil, React, Phonegap, JavaScript.}

\newpage
\large{\textbf{Abstract}}\\

The recent emergence of smartphones has led to a XX in the amount of mobile applications available for all existing platforms. In the app markets a wide variety of applications can be found, ranging from games or social networks to some more career-oriented. The result is a rapidly growing market devoted to their development with programming languages such as Java, Swift or Objective C.

However, very different platforms exist, owned by different companies. This means that once an app has been developed for one of these platforms, in order to run it on another one, the application has to be completely doveloped almost from scratch again.

In order to ease the task of porting an app to a different computing environment emerged the concept of hybrid mobile applications. Using web technologies such as HTML5, CSS and JavaScript, and an API, cross-platform mobile applications can be developed in a a quick and efficient manner.

Currently, there are two technologies for the development of hybrid mobile applications: Phonegap, used by a broad amount of developers, and the more recent React Native, which is still under development. In this work, the biggest and most important differences between both technologies will be explained, and in order to hightlight them, the same application will be developed for them both.

A React Native test scenario will be designed and implemented as well, to act as an introduction to this technology. Every step of the process will be thoroughly documented, from  installing the required software to running the app on a device.

\vspace{5mm}
\large{\textbf{Keywords: } Aplicación Móvil, React, Phonegap, JavaScript.}
\afterpage{\blankpage}

\newpage
\pagenumbering{gobble}
\tableofcontents
\clearpage
\pagenumbering{arabic}

\chapter{Introduction}
\label{ch:intro}
\justifying
\section{Context}
Not long ago, before developing a mobile application, or app, one of the most important decisions that had to be made was choosing the platforms that your app would run on. Because a different programming language is used for each platform, the same application would have to be developed twice, thrice or even more times. In fact, the amount of work involved could be such that would limit the efforts of the development team to a single platform.

Along with mobile applicationss, another way of adapting web services for smartphones emerged: web mobile applications. This type of applications use standard web technologies, generally HTML5, CSS and JavaScript, to adapt web pages and make them suitable for the smaller screens of smartphones and tablets. Though this concept saves time, some vital limitations remain, mostly access to native device functionalities such as camera or GPS.

Aiming to increase the productivity and efficiency of a development process, the concept of hybrid mobile applications was born. Through an Aplication Programming Interface, or API, it is possible to embed HTML5 apps inside a native container, so the same code can be ported across systems. Hybrid mobile applications combine the best elements of native and HTML5, and therefore represent a new stage in the evolution of mobile applications development.

In this context, currently the most popular technology for this purpose is Adobe PhoneGap~~\cite{phonegap}. However, some of its own peculiarities make PhoneGap apps inefficient and slow, and somehow feel less polished  than a native app. In an attempt to fix these issues and challenge PhoneGap's dominant position, React Native~\cite{reactnative} came on the scene on April 2015. React Native's main feature is that it uses native components to achieve a completely native application user experience.

\section{Project goals}

This project aims to provide an explanation on two main topics: firstly, the differences betweeen  PhoneGap and React Native, and secondly, a test scenario to help understand the most important features that revolve around React Native.

Among the main goals inside this project, we can find:

\begin{itemize}  
	\item Comparing current mobile application development options: their advantages and disadvantages will be exposed and detailed.
	\item Understanding what React Native: what it is and how it works, its architecture, runtime execution and lifecycle, among others.
	\item Design and build a simple test scenario to demonstrate React Native's capabilities.
\end{itemize}

\section{Structure of this document}

This project is divided in X chapters, arranged following a general-to-specific pattern. The structure is as follows:

\textit{Chapter 1:} provides an introduction to the project, explains the main goals to be achieved and a strctural overview of this document.

\textit{Chapter 2:} describes the main options currently available for developing a mobile application, and their advantages and disadvantages.

\textit{Chapter 3:} brief introduction to PhoneGap in order to fully understand how it works and distinguish it from React Native.

\textit{Chapter 4:} description of React~\cite{react}, a JavaScript library for creating the user interface in a Model-View-Controller (MVC) architecture based web page.

\textit{Chapter 5:} a detailed description of React Native and its motivation, architecture, runtime execution, and other relevant features.

\textit{Chapter 6:} description and implementation of a test scenario with detailed explanations. This Chapter aims to serve as an introduction to the development of mobile applications using React Native and flatten the learning curve. It is not a showcase of all React Native features and capabilities, but will serve as a solid foundation for those with real interest in this technology.

\textit{Chapter 7:} this paper ends with a reflection on React Native and the developer experience, summarizing the achieved goals and providing future lines of work.
\chapter{Understanding Mobile Application Development Options}
\justifying
\section{Introduction}

This chapter provides a detailed explanation on the distinct types of mobile applications that have been already introduced. As a result, the reader will be aware of the pros and cons that every scenario involves. Special attention will be payed to hybrid mobile applications, not only because the whole project is based on them, but also because they are the less well-known.

One section will be devoted for each and every scenario, discussing how are they developed and exposing their strengths and weaknesses. A brief summary of the technologies and tools needed for each will also be provided. The chapter will conclude with a table to sum up all the available scenarios.

\section{Native Mobile Applications}

A native mobile application is an application coded in a specific programming language. The most popular are Java for Android and Swift or Objective C for iOS operating systems. They provide the best usability, the best features, the best performance, the best overall user experience, plus are highly reliable. Because of this, apps like video games tend to be developed this way, as they are resource intensive and a bad performance could hurt the user experience. However, this type of app is tied to one single operating system, forcing the development team to make duplicate versions that work on other platforms. Therefore this may not be the best option for apps intended to spread throughout the whole market.

Their key features are:

\begin{itemize}
	\item \textbf{Multi touch handling:} native apps support all kinds of multi touch events, from double taps or pinches to pressure-sensitive taps.
	\item \textbf{Built-in components:} camera, GPS or encrypted storage are only a few of the features native to a device that can be smoothly integrated into an app. Built-in components are of such importance to some apps that may be a deciding factor on which mobile technology one should choose.
	\item \textbf{Fast graphics and fluid animations:} for those apps which are using a lot of data and require a fast refresh, are highly interactive or make use of intensely computational algorithms, native apps provide the highest efficiency and the best results.
	\item \textbf{High reliability:} the abstraction while coding is lower than on the other types of mobile applications. Therefore, as a developer, you got more control over what is going on inside your app, and nasty surprises are less likely to happen. 
	\item \textbf{Improved user experience and familiarity:} people are accustomed to the native platform. Making use of all the native features will be easier for the average user and the app will be just plain easier to use.
	\item \textbf{Documentation:} apart from the official docuementation, there are thousands of books for Android and iOS development, and many more blogs, articles and websites with detailed information about every feature and little crevice.
\end{itemize}

After taking a look at the pros of developing a native app, it is clear that they are unbeatable on some aspects. However, they suffer from a number of issues:

\begin{itemize}
	\item \textbf{System restricted:} the fact that native apps are developed using specific programming languages turns porting an app between platforms into a time-consuming task because most of the code will be simply useless. 
	\item \textbf{Difficult to develop:} the programming languages used are not the easiest ones, so simply cutting and pasting Objective-C or Java will neither suffice nor work.
	\item \textbf{High level of experience required:} because native apps are difficult to develop, the level of experience required is higher than other development scenarios. Indeed, the technological know-how of a developer is a very important consideration.
	\item \textbf{Require an integrated development environment:} an integrated development environment, or IDE, provides tools for building, debugging, managing, version control and some other critical tools that professional developers need. These tools are needed because of the difficulties involved in the development process. In fact, usually different IDEs are used for each programming language, and even though they tend to be very similar to one another, so they become another thing to add to the list of inconvenients.
\end{itemize}

After considering all the pros and cons concerning the development of a native mobile application, we can conlude that it is the best way to develop ambitious, compute-intensive, single platform mobile applications.

\section{Web Mobile Applications}

A web mobile application, also known as HTML5 mobile application, is basically a web page designed to work on the small screen of a smartphone. This makes them compatible with every modern mobile browser. Developing them is easier as well: the technological bar and the learning curve of website programming languages are singnificantly lower than those of Objective C, for example.

Web apps most remarkable aspects are the following:

\begin{itemize}
	\item \textbf{Write once, run anywhere:} web applications take this concept to the extreme. As long as the device is running a modern browser, the user can use the app. This makes sure that it will reach the whole market. 
	\item \textbf{Searchable:} because the content is on the web, it can be searched, which can be a huge boost to its popularity.
	\item \textbf{Easy to develop:} the complexity involved in developing a web page is usually diminished by the fact that is easier to understand and of a higher level than native apps programming languages.
	\item \textbf{Distribution and support:} native mobile applications are dependent upon marketplaces for their distribution and support. Nevertheless, web applications are hosted on a server kept under control by the development team, so fixing a bug or adding features becomes a much easier task.
\end{itemize}



\end{document}